\chapter{Installation}
\section{Einführung}
LaTeX selbst ist keine Applikation wie MS Worls aber es ist ein \textbf{Softwarepacket} für die Textverarbeitung. Ein Benutzer arbeitet nur mit Textdateien. Die Formatierung führ man durch den Kommands. Um eine Literaturform zu erhalten, muss der Text kompilliert werden. Deswegen gibt es keine Applikation mit dem Namen LaTeX aber man kann über LaTex wie über eine Programmiersprache nachdenken. Es gibt die Applikationen wie z.B. MikTeX die einen Latex und einen Kompiler dafür enthalten. MikTeX selbs hat aber nicht gute GUI, deswegen muss man noch einen GUI wie z.B. TeXnicCenter installieren. Andere Möglichkeit für Windows ist TexMaker, der Latex, Kompiler und auch GUI enthält. 

Um MiKtex zu installieren empfehlt man zuerst die specifischen Build y.b. von einer Uniresität herunterzuladen und dann es aus diesem Datei installieren. Es ist gut (auf Tschechisch) beschereibt in:
https://latex.fekt.vut.cz/instalace-miktex-a-spol-pc-windows/instalace-miktex/

Konfiguration von PDF Viewer in TeXnicCe
\chapter{Writing standards}
čšřý

Česká republika
Definiton of text parameters is done with help of commands. Commands are defined with \textbackslash after which the name of command follows \textcolor {blue}{\bf\textbackslash commandAK\bf}. There is no space between "'\textbackslash"'and name of command. After that often often follows parameters in curly brackets = braces. Some commands have more parameters in more curly brackets, some have one curly bracket and some parameters in square brackets. 

La